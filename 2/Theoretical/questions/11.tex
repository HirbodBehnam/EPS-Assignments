\smalltitle{سوال 11}\\
هدف ما در این سوال بدست آوردن
$P(X = k)$
است که بفهمیم از چه توزیعی پیروی می‌کند.
برای شروع فرض کنید که
$m$
کشتی می‌خواهند که از بندرگاه خارج شوند و تعداد کل کشتی‌های موجود در بندرگاه برابر
$n$
است. طبق گفته‌ی سوال متغیر تصادفی تعداد کشتی‌هایی که در بندرگاه هستند را اگر
$Y$
بگیریم آنگاه
$Y \sim Poiss(4)$
است. پس احتمال اینکه
$n$
کشتی در بندرگاه باشند برابر
$P(Y = n)$
است. همچنین طبق گفته‌های سوال احتمال اینکه از
$n$
کشتی،
$m$تای
آن‌ها خارج شوند برابر است با
${n \choose m} 0.6^m ~ 0.4^{n-m}$.
حال دقت کنید که برای محاسبه‌ی
$P(X = m)$
ما تعداد کشتی‌هایی که قرار است در بندر بمانند را باید از 0 تا بی‌نهایت تغییر دهیم و احتمال مطلوب آن حالت را
حساب کنیم و در نهایت همه را با هم جمع بزنیم. برای همین داریم:
\begin{gather*}
    P(X = m) = \sum_{i=0}^{\infty} P(Y=m+i) {m + i \choose m} 0.6^m ~ 0.4^i
    = \sum_{i=0}^{\infty} \frac{4^{m+i} e^{-4}}{(m+i)!} \frac{(m+i)!}{m! i!} 0.6^m ~ 0.4^i\\
    = \frac{4^m e^{-4} 0.6^m}{m!} \sum_{i=0}^{\infty} \frac{0.4^i 4^i}{i!}
    \stackrel{\text{Taylor}}{=} \frac{2.4^m e^{-4}}{m!} e^{1.6} = \frac{2.4^m e^{-2.4}}{m!} \sim Poiss(2.4)
\end{gather*}