\smalltitle{سوال 8}\\
\smalltitle{الف}\\
با توجه به داده‌های سوال
\lr{PDF}
زمان واکسیناسون هر فرد برابر
$\frac{1}{\lambda} e^{\frac{-x}{\lambda}}$
است. حال متغیر‌های زیر را تعریف می‌کنیم:
\[
    \begin{array}{ll}
        T_1: & \text{زمانی که فردی که در باجه‌ی اول است صرف می‌کند}\\
        T_2: & \text{زمانی که فردی که در باجه‌ی دوم است صرف می‌کند}\\
        T_m: & \text{زمانی که واکسیناسون ما طول می‌کشد}
    \end{array}
\]
پس باید احتمال
$P(T_1 > T_2 + T_m) = P(T_1 > T_2) P(T_1 > T_2 + T_m | T_1 > T_2)$
را باید بدست آوریم.
در ابتدا احتمال
$P(T_1 > T_2)$r
را بدست می‌آوریم:
(مسئله را فعلا برای حالت کلی $\lambda$ حل می‌کنیم.)
% https://math.stackexchange.com/a/1332436/424863
\begin{gather*}
    P(T_1 > T_2) = \int_{0}^{\infty} f_X(x) P(T_1 > x) dx = \int_{0}^{\infty} \lambda e^{-\lambda x} (1-e^{-\lambda x}) dx
    = (-e^{- \lambda x} + 0.5e^{-2 \lambda x}) \big |_0^{\infty} = 0.5
\end{gather*}
حال
\lr{memoryless}
بودن توزیع نمایی را اثبات می‌کنیم. در ابتدا فرض می‌کنیم
$P(X > t) = S(t)$
است. پس
$S(t) = 1 - (1 - e^{-\lambda t}) = e^{-\lambda t}$
حال دقت کنید که
$S(t) = S(1)^t = S(1) \times \cdots \times S(1)$
است. این نتیجه می‌دهد که
$S(a + b) = S(a)S(b)$
است. پس داریم:
$\frac{S(a+b)}{S(a)} = S(b) \implies \frac{P(X > a+b)}{P(X > a)} = P(X > a + b | X > a) = P(X > b)$
پس با توجه به این موضوع، طبق احتمال
$P(T_1 > T_2)$
می‌توانیم بنویسیم:
\begin{gather*}
    P(T_1 > T_2 + T_m | T_1 > T_2) = P(T_1 > + T_m) = 0.5\\
    \implies P(T_1 > T_2 + T_m) = P(T_1 > T_2) P(T_1 > T_2 + T_m | T_1 > T_2) = 0.5 \times 0.5 = 0.25
\end{gather*}
\smalltitle{ب}\\
% https://llc.stat.purdue.edu/2014/41600/notes/prob3205.pdf
در ابتدا باید
$\mathbb{E} [\min (T_1, T_2)]$
را بدست بیاوریم. برای این کار ابتدا
$\operatorname{CDF} (\min (T_1, T_2))$
را بدست می‌آوریم. برای این کار عدد
$a > 0$
را انتخاب می‌کنیم.
\begin{gather*}
    P(\min (T_1, T_2) > a) = P(T_1 > a, T_2 > a) \stackrel{\text{independence}}{=} P(T_1 > a) P(T_2 > a)\\=
    (1 - (1 - e^{-a\frac{1}{\lambda}})) (1 - (1 - e^{-a\frac{1}{\lambda}})) = e^{-2a\frac{1}{\lambda}}
    \sim \operatorname{Exp}(\frac{2}{\lambda})
\end{gather*}
پس طبق خواص توزیع نمایی می‌دانیم که
$\mathbb{E} [\min (T_1, T_2)] = \frac{\lambda}{2}$
است. همچنین امیدریاضی زمان واکسیناسون خود ما
$\lambda$
است. پس مدت زمانی که در مرکز هستیم برابر است با
$\frac{\lambda}{2} + \lambda = \frac{3\lambda}{2}$


