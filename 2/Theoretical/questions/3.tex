\smalltitle{سوال 3}\\
احتمال خرید از پیتزا فروشی B را
$P_B$
می‌نامیم و احتمال رسیدن پیتزا بعد از حداکثر
$i$
دقیقه را
$T_i$
می‌نامیم.
باید
$P(P_B|T_{40})$
را بدست آوریم.
طبق قانون بیز داریم:
\begin{gather*}
    P(P_B|T_{40}) = \frac{P(T_{40}|P_B) P(P_B)}{P(T_{40})} = \frac{P(T_{40}|P_B) P(P_B)}{P(T_{40}|P_A) P(P_A) + P(T_{40}|P_B) P(P_B)}
\end{gather*}
حال هر کدام از قسمت‌ها را حساب می‌کنیم. می‌دانیم که
$P(P_A) = \frac{2}{5}, P(P_B) = \frac{3}{5}$
است. حال احتمال زودتر رسیدن پیتزا از پیتزا فروشی
$A$
را باید حساب کنیم. این پیتزا فروشی 30 دقیقه که همیشه آماده کردن پیتزا برایش طول می‌کشد. همچنین
PDF
تحویل غذای آن برابر
$0.1e^{-0.1x}$
است.
(دقت کنید که میانگین 10 بود و می‌دانیم
$\mathbb{E} [X] = \frac{1}{\lambda}$)
پس احتمال اینکه غذا زودتر از 10 دقیقه تحویلش طول بکشد برابر است با:
\begin{gather*}
    \int_0^{10} 0.1e^{-0.1x} dx = -e^{-0.1x} \big |_0^{10} = 1 - \frac{1}{e}
\end{gather*}
پس داریم:
\begin{gather*}
    P(P_A) P(T_{40}|P_A) = \frac{2}{5} (1 - \frac{1}{e}) \approx 0.252
\end{gather*}
همچنین این عبارت را برای پیتزا فروشی
$B$
حساب می‌کنیم. زمان تحویل این پیتزا فروشی همیشه برابر 15 دقیقه است. پس باید حساب کنیم که به چه احتمالی
پیتزا زودتر از 25 دقیقه آماده می‌شود. پس داریم:
\begin{gather*}
    \int_0^{25} \frac{1}{5\sqrt{2\pi}}e^{\frac{-(x-30)^2}{50}} dx \approx 0.152
\end{gather*}
پس در نتیجه:
\begin{gather*}
    P(P_B) P(T_{40}|P_B) = \frac{3}{5} \times 0.152 = 0.0912
\end{gather*}
پس از قانون بیز داریم:
\begin{gather*}
    P(P_B|T_{40}) = \frac{0.0912}{0.0912 + 0.252} = 0.265
\end{gather*}

