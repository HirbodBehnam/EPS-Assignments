\smalltitle{سوال 4}\\
\smalltitle{الف}\\
باید عملا
$P(X + Y = k)$
را بدست بیاوریم. برای این کار از قانون احتمال کل و شرطی استفاده می‌کنیم:
\begin{gather*}
    P(X + Y = k) = \sum_{i=0}^k P(X + Y = k | X = i) P(X = i) = \sum_{i=0}^k P(Y = k - i | X = i) P(X = i)\\
    \stackrel{\text{Independence}}{=} \sum_{i=0}^k P(Y = k - i) P(X = i) = \sum_{i=0}^k \frac{\lambda_2^{k-i} e^{-\lambda_2}}{(k-i)!} \frac{\lambda_1^k e^{-\lambda_1}}{k!} \\
    = e^{-(\lambda_1 + \lambda_2)} \sum_{i=0}^k \frac{\lambda_2^{k-i} \lambda_1^k}{k!(k-i)!} = \frac{e^{-(\lambda_1 + \lambda_2)}}{k!} \sum_{i=0}^k {k \choose i} \lambda_2^{k-i} \lambda_1^k
    = \frac{e^{-(\lambda_1 + \lambda_2)} (\lambda_1 + \lambda_2)^k}{k!}
\end{gather*}
پس در نتیجه
$X + Y \sim Poiss(\lambda_1 + \lambda_2)$
است.
\\\smalltitle{ب}\\
% https://www.youtube.com/watch?v=DXkoD0MaW_U
دقت کنید که در ابتدا احتمال اینکه در 20 دقیقه‌ی اول کسی وارد شیرینی فروشی نشده باشد با اینکه در 10 دقیقه‌ی
بعد تنها یک نفر وارد شیرینی فروشی شده باشد مستقل است. پس عملا باید احتمال این را حساب کنیم که در 10 دقیقه
تنها یک نفر وارد شیرینی فروشی شده باشد. پس طبق قسمت قبل متغیر تصادفی
$Z$
که نشان دهنده‌ی وارد شدن
$x$
نفر در ده دقیقه است از توضیح
\lr{poisson}
با
$\lambda = \frac{10}{60} (10 + 20) = 5$.
پس در نتیجه
\begin{gather*}
    P(Z = 1) = \frac{5 e^{-5}}{1!} = 5 e^{-5}
\end{gather*}