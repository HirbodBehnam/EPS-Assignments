\smalltitle{سوال 2}\\
\smalltitle{الف}\\
می‌دانیم که
$\int_{-\infty}^{+\infty} f_x dx = 1$
است. پس می‌توان از این جا
$A$
را بدست آورد.
\begin{gather*}
    \int_{-\infty}^{+\infty} f_X(x) = \int_{-\infty}^{+\infty} A e^{-2|x|} dx + \int_{-\infty}^{+\infty} \frac{2}{\sqrt{\pi}} e^{-4x^2} dx = 1
\end{gather*}
% https://math.stackexchange.com/a/886561
به کمک دانش ریاضی 2 جواب
$I = \int_{-\infty}^{+\infty} e^{-4x^2} dx$
را بدست می‌آوریم.
\begin{gather*}
    I^2 = \int_{-\infty}^{+\infty} e^{-4x^2} dx \times \int_{-\infty}^{+\infty} e^{-4y^2} dy =
    \int_{-\infty}^{+\infty}\int_{-\infty}^{+\infty} e^{-4(x^2 + y^2)} dx ~ dy
\end{gather*}
حال از تغییر مختصات قطبی استفاده می‌کنیم و
$x^2+y^2=r^2, dx ~ dy = r ~ d \theta ~ dr $
قرار می‌دهیم:
\begin{gather*}
    I^2 = \int_{0}^{2\pi} \int_{0}^{+\infty}  r e^{-4r^2} dr ~ d \theta = 2\pi \int_{0}^{+\infty}  r e^{-4r^2} dr
    \stackrel{u=r^2, du = 2r ~ dr}{=}  \pi \int_{0}^{+\infty}  e^{-4u} du =  \pi \frac{-1}{4} (e^{-\infty} - e^0) = \frac{\pi}{4}
    \\
    I = \frac{\sqrt{\pi}}{2}
\end{gather*}
پس
$\int_{-\infty}^{+\infty} \frac{2}{\sqrt{\pi}} e^{-4x^2} dx = 1$
است و باید
$\int_{-\infty}^{+\infty} A e^{-2|x|} dx = 0$
باشد. پس
$A=0$
است.

حال
$\mathbb{E}[X] = \int_{-\infty}^{+\infty} x f_X(x) dx = \frac{2}{\sqrt{\pi}} \int_{-\infty}^{+\infty} x e^{-4x^2} dx$
را حساب می‌کنیم.
حال اگر دقت کنید متوجه می‌شوید که این تابع یک تابع فرد است! پس انتگرال منفی تا مثبت بی‌نهایت آن برابر 0 است.
پس
$\mathbb{E}[X] = 0$
است. پس داریم (اکثرا با توجه به قسمت قبل):
\begin{gather*}
    \operatorname{Var}(X) = \mathbb{E}[X^2] = \int_{-\infty}^{+\infty} x^2 f_X(x)
    = \frac{2}{\sqrt{\pi}} \int_{-\infty}^{+\infty} x \times (x e^{-4x^2})\\
    = \frac{2}{\sqrt{\pi}} (x (\frac{-e^{-4x^2}}{8}) \big |_{-\infty}^{+\infty} - \int_{-\infty}^{+\infty} \frac{-e^{-4x^2}}{8} dx)
    = \frac{2}{\sqrt{\pi}} (0 + \frac{1}{8} \frac{\sqrt{\pi}}{2})\\
    \implies \mathbb{E}[X^2] = \operatorname{Var}(X) = \frac{1}{8}
\end{gather*}
\smalltitle{ب}\\
در کلاس حل تمرین یاد گرفتیم:
\begin{gather*}
    f_Y(y) = \sum f_X(g_i^{-1}(y)) |\frac{d}{dy} g_i^{-1}(y)|
\end{gather*}
عملا کاری که باید انجام بدهیم این است که تابع تبدیل کننده‌ی متغیر تصادفی
$X$ به $Y$
که همان
$g(x)$
است را به بازه‌های اکیدا یکنوا بشکنیم و سپس آن‌ها را در این فرمول جایگزین کنیم.
پس بازه‌ی
$(-\infty, +\infty)$
را به صورت زیر افراز می‌کنیم:
(دقت کنید که در بازه‌ی
$[-0.5, 0.5]$
تابع اکیدا یکنوا نیست.)
% http://cknudson.com/StudentWork/Transforms.pdf
\[
    \begin{array}{lll}
       A_0 = (-\infty, -0.5) & g_1(x) = \sqrt{-2x - 1} & g_1^{-1}(x) = -\frac{x^2 + 1}{2}\\
       A_1 = (0.5, \infty) & g_2(x) = \sqrt{2x - 1}  & g_2^{-1}(x) = \frac{x^2 + 1}{2}
    \end{array}
\]
پس داریم:
\begin{gather*}
    f_Y(y) = \frac{2}{\sqrt{\pi}}e^{-4(\frac{y^2 + 1}{2})^2} |y|
    + \frac{2}{\sqrt{\pi}}e^{-4(-\frac{y^2 + 1}{2})^2} |-y|
    = \frac{4}{\sqrt{\pi}}e^{-(y^2 + 1)^2} |y|
\end{gather*}


