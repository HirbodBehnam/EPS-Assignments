\smalltitle{سوال 5}\\
\smalltitle{الف}\\
در ابتدا حساب می‌کنیم که در یک پرتاب به چه احتمالی حداقل دو سکه رو می‌آیند. برای این کار کافی است
که حالت بندی کنیم.
\begin{gather*}
    \frac{1}{2} \frac{1}{4} \frac{5}{6} + \frac{1}{2} \frac{3}{4} \frac{1}{6} + \frac{1}{2} \frac{1}{4} \frac{1}{6} + \frac{1}{2} \frac{1}{4} \frac{1}{6}
    = \frac{10}{48} = \frac{5}{24}
\end{gather*}
حال برای برنده شدن 80 تومان حالت بندی می‌کنیم:
\begin{gather*}
    {10 \choose 8} (\frac{5}{24})^8 (1-\frac{5}{24})^2 + {10 \choose 9} (\frac{5}{24})^9 (1-\frac{5}{24})^1 + {10 \choose 10} (\frac{5}{24})^{10} (1-\frac{5}{24})^0
\end{gather*}
\smalltitle{ب}\\
متغیر تصادفی
$X_i$
را به این صورت تعریف می‌کنیم که آیا در دست
$i$ام
حداقل دو رو می‌آید یا خیر؟ در این صورت مقدار متغیر تصادفی را 10 قرار می‌دهیم و در غیر این صورت 0.
حال داریم:
\begin{gather*}
    \mathbb{E} [X] = \sum_{i=1}^{10} \mathbb{E} [X_i] = \sum_{i=1}^{10} 10 \frac{5}{24} = 100 \frac{5}{24}
\end{gather*}
و همچنین مشخص است که:
\begin{gather*}
    \mathbb{E} [X^2] = \sum_{i=1}^{10} \mathbb{E} [X_i^2] = \sum_{i=1}^{10} 10^2 \frac{5}{24} = 1000 \frac{5}{24}
\end{gather*}
پس:
\begin{gather*}
    \operatorname{Var}(X) = \mathbb{E} [X^2] - \mathbb{E} [X] = 1000 \frac{5}{24} - 100 \frac{5}{24} = 900 \frac{5}{24}
\end{gather*}
\smalltitle{ج}\\
دقت کنید که مثل این است که چهار دست دیگر بازی کنیم و می‌خواهیم بدانیم به چه احتمالی حداقل یک دست می‌بریم.
پس داریم:
\begin{gather*}
    \sum_{i=1}^4 {4 \choose i} (\frac{5}{24})^{i} (1-\frac{5}{24})^{4-i}
\end{gather*}
\smalltitle{د}\\
در ابتدا احتمال همه‌ی سکه‌ها رو بیایند را حساب می‌کنیم و با توجه به اصل متمم احتمال حداقل یک سکه پشت را حساب می‌کنیم.
\begin{gather*}
    P = 1 - \frac{1}{2} \frac{1}{4} \frac{1}{6} = \frac{47}{48}
\end{gather*}
اول از همه دقت کنید که وقتی که در بازی می‌بازیم ولی دست بعدش را بازی می‌کنیم دقیقا مثل این است که یک دست
بازی جدید را شروع کرده‌ایم! حال با این مثال فرض می‌کنیم که قرار است
$x$
بار بازی کنیم و به احتمال
$(\frac{47}{48})^x$
تمام بازی‌ها را می‌بریم. دقت کنید که با برد همه‌ی این بازی‌ها
$10x$
واحد پول بدست می‌آوریم. پس امیدریاضی پولی که پس از
$x$
بار بازی کردن بدست می‌آوریم
$10x(\frac{47}{48})^x$
است. پس حداکثر این تابع را بدست می‌آوریم:
\begin{gather*}
    \frac{d}{dx} 10x(\frac{47}{48})^x = 10 (\frac{47}{48})^x + 10x(\frac{47}{48})^x \ln \frac{47}{48} = 0\\
    \implies (\frac{47}{48})^x \times (1 + x \ln \frac{47}{48}) = 0\\
    \implies 1 + x \ln \frac{47}{48} = 0 \implies x = \frac{-1}{\ln \frac{47}{48}} \approx 47.5
\end{gather*}
پس ماکسیموم بین امیدریاضی بین
$x=47$
و
$x=48$
را بدست می‌آوریم:
\begin{gather*}
    x = 47 \implies 470 (\frac{47}{48})^{47} \approx 174.7\\
    x = 48 \implies 480 (\frac{47}{48})^{48} \approx 174.7
\end{gather*}
پس این دو عدد برابر است و او یا 47 و یا 48 بار اگر بازی کند حداکثر پول را بدست می‌آورد.

