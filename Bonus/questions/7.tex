\smalltitle{سوال 7}\\
% https://math.stackexchange.com/a/3427113/424863
فرض کنید که
$a$
توپ در جعبه وجود دارد. مشخص است که
$a$ بر $m$
بخش پذیر است و برای همین احتمال انتخاب هر نوع برابر است. ما به دنبال
$\mathbb{E}(X)$
هستیم که
$X$
متغیر تصادفی تعداد توپ‌ها با نوع‌های متفاوت است. حال از خواص امیدریاضی استفاده می‌کنیم و داریم:
\begin{gather*}
    \mathbb{E}(X) = \sum_{i=1}^{m} \mathbb{E}(X_i) = m\mathbb{E}(X_1)
\end{gather*}
در اینجا
$\mathbb{E}(X_i)$
نشان دهده‌ی امیدریاضی آمدن توپ نوع
$i$ام است.
(به کمک
\lr{indicator variable})
همچنین دقت کنید که تمامی
$\mathbb{E}(X_i)$ها
با هم برابر‌اند چرا که احتمال انتخاب شدن توپ‌ها برابر است.
برای سادگی ما امیدریاضی نیامدن توپ نوع
$i$
را حساب می‌کنیم و جواب حاصل را از یک کم می‌کنیم.
\begin{gather*}
    P(X_1) = 1 - \frac{{a - a/m \choose n}}{{a \choose n}}
\end{gather*}
پس جواب نهایی سوال برابر است با
\begin{gather*}
    \mathbb{E}(X) = m (1 - \frac{{a - a/m \choose n}}{{a \choose n}})
\end{gather*}
که
$a$
تعداد کل توپ‌های درون جعبه است.