\smalltitle{سوال 2}\\
% https://www.youtube.com/watch?v=9XK61C_KpJk
در ابتدا دقت کنید که هر حرکت نسبت به مرحله‌ی قبل مستقل است.
همچنین برای مدل سازی سوال فرض کنید که متغیری داریم که از صفر شروع می‌شود و با احتمال
$p$
منهای یک می‌شود و با احتمال
$1-p$
بعلاوه‌ی یک می‌شود. می‌خواهیم ببینیم این متغیر با چه احتمالی
1
می‌شود. قرارداد می‌کنیم که
$P_i$
برابر است با احتمال رسیدن به عدد
$i$
است وقتی که از 0 شروع کنیم.
ابتدا توجه کنید که با توجه به قانون احتمال کل داریم:
\begin{gather*}
    P_{1} = (1-p) \times 1 + p \times P_{2}
\end{gather*}
این عبارت بدین معنا است که اگر در مرحله‌‌ی اول عدد منهای یک شد که با احتمال 1 به جواب می‌رسیم. در غیر این صورت
(که احتمال رخ دادن آن
$p$
است)
با احتمال
$P_{2}$
به
$P_{1}$
می‌رسیم. دلیل این موضوع این است که هر عدد از هم مستقل است. پس وقتی می‌خواهیم از
1-
به
1
برسیم مثل این است که از 0 به
2 برسیم.
حال دقت کنید که به خاطر همین مستقل بودن داریم
$P_i = P_1^i$. پس معادله‌ی ما برابر است با
\begin{gather*}
    P_{1} = 1-p + p \times P_{1}^2 \implies p \times P_{1}^2 - P_{1} + 1-p\\
    \implies P_{1} = \frac{1 \pm \sqrt{1 - 4p(1-p)}}{2p} = \frac{1 \pm (2p - 1)}{2p}\\
    \implies P_{1} = 1 \quad \text{or} \quad \frac{1-p}{p}
\end{gather*}
حال مفهوم دو جواب چیست؟ در ابتدا دقت کنید که
$0 \leq p \leq 1$
است و همچنین
$0 \leq P_{1} \leq 1$.
حال دقت کنید که اگر
$0 \leq p \le 0.5$
باشد آنگاه
$1 < \frac{1-p}{p}$
می‌شود و در شرط
$0 \leq P_{1} \leq 1$
صدق نمی‌کند.
پس جواب مسئله را به این صورت اعلام می‌کنیم:
\begin{gather*}
    P_{1} = \begin{cases}
        1 & 0 \leq p \le 0.5\\
        \frac{1-p}{p} & 0.5 \leq p \leq 1\\
    \end{cases}
\end{gather*}
