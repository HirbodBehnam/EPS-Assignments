% https://sphweb.bumc.bu.edu/otlt/MPH-Modules/PH717-QuantCore/PH717-Module6-RandomError/PH717-Module6-RandomError11.html
\smalltitle{سوال 7}\\ % c(152, 148, 153, 153)
\smalltitle{الف}
از آنجایی که تعداد داده‌ها کوچک است از
\lr{t-test}
استفاده می‌کنیم. دقت کنید که در صورتی که مثلا می‌خواستیم از
\lr{CLT}
استفاده کنیم، بازه‌ی اطمینان برابر بود با
$\overline{X} \pm z_{\frac{\alpha}{2}} \frac{\sigma}{\sqrt{n}}$.
ولی در این سوال از آنجا که تعداد داده‌ها کم است از
\lr{t-distribution}
استفاده می‌کنیم و بازه‌ی اطمینان برابر می‌شود با
$\overline{X} \pm t_{\frac{\alpha}{2}, n - 1} \frac{\sigma}{\sqrt{n}}$.
به کمک
R
یا
\lr{t-table}
مقدار
$t_{\frac{\alpha}{2}, n - 1} = t_{0.025, 3}$
را حساب می‌کنیم که برابر است با
\lr{-3.182}. % qt(0.025,3)
حال بازه‌ی اطمینان را بدست می‌آوریم.
\begin{gather*}
    \left[ \overline{X} + t_{\frac{\alpha}{2}, n - 1} \frac{\sigma}{\sqrt{n}} , \overline{X} - t_{\frac{\alpha}{2}, n - 1} \frac{\sigma}{\sqrt{n}} \right]\\
    \left[ 151.5 - 3.182 \times \frac{2.38}{2} , 151.5 + 3.182 \times \frac{2.38}{2} \right]\\
    \left[147.713, 155.287 \right]
\end{gather*}
% t.test(c(152,148,153,153), mu = 150, alternative = "greater")
\smalltitle{ب}
فرض صفر این است که میانگین زمان تمرین‌ها کمتر از 150 باشد و فرض دیگر این است که بزرگتر از 150 باشد.
حال مقدار
$t$
را برای آماره حساب می‌کنیم.
\begin{gather*}
    t = \frac{\mu - \mu_0}{\frac{\sigma}{\sqrt{n}}} = \frac{151.5 - 150}{1.19} = 1.26
\end{gather*}
حال از آنجا که
$1.26 < 3.182$
است پس نمی‌توان فرض صفر را رد کرد. یعنی اینکه نمی‌توان گفت که تمرین‌ها ساده بودند.
(عدد
$3.182$
را از قسمت قبل داریم.)
\\\smalltitle{ج}
اگر سطح اهمیت را کاهش دهیم، احتمال خطای نوع 1 کاهش می‌یابد. دقت کنید که با کاهش این مقدار عملا رد کردن
$H_0$
راحت‌تر می‌شود.
اما برعکس وقتی این عدد کم می‌شود احتمال خطای نوع دو افزایش می‌یابد.