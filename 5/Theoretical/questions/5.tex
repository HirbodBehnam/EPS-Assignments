\smalltitle{سوال 5}\\
\smalltitle{الف}
\begin{gather*}
    H_0: \mu = \mu_0\\
    H_1: \mu > \mu_0\\
    \overline{X} = 6730 \quad \sigma = 900 \quad n = 150\\
    W(X_i) = \frac{\overline{X}-\mu_0}{\frac{\sigma}{\sqrt{n}}} = \frac{6730 - 6600}{\frac{900}{\sqrt{150}}} \approx 1.77\\
    \alpha = 1 - \Phi(1.77) \approx 1 - 0.96 = 0.04
\end{gather*}
\smalltitle{ب}
از آنجایی که
$(1 - 0.95) > 0.04$
است، پس
$H_0$
را رد می‌کنیم. یعنی اینکه این ادعای سلف درست نیست.
\\\smalltitle{ج}
از آنجایی که
$(1 - 0.99) < 0.04$
است، پس
$H_0$
را نمی‌توان رد کرد. دقت کنید که این به این معنا نیست که
$H_1$
درست است. بلکه بدین معنا است که 
\\\smalltitle{د}
تنها در حساب کردن
$\alpha$
کمی تفاوت حاصل می‌شود. دقت کنید که در قبل
$\alpha = P(X > W(X_i)) ~ X \sim N(0, 1)$
بود. ولی حالا که دو طرفه است، باید
$\alpha = P(|X| > W(X_i)) ~ X \sim N(0, 1)$
را پیدا کنیم. دقت کنید که متغیر تصادفی
$X$
نسبت به محور
$y$
تقارن دارد. پس 
$\alpha = P(|X| > W(X_i)) = 2P(X > W(X_i)) = 0.08$
است. با توجه به این
\lr{p-value}
جفت فرض صفر قسمت ب و ج رد می‌شود.




