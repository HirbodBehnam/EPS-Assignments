\smalltitle{سوال 3}\\
در ابتدا دقت کنید که کسر حاصل از تقسیم نقاط درون دایره به کل نقاط برابر
$\frac{\pi}{4}$
است. پس دقت کنید که خطا برای
$\frac{\pi}{4}$
باید کمتر از
$0.0025$
باشد تا خطا برای
$\pi$
کمتر از
$0.01$
باشد.
حال متغیر تصادفی
$X$
را به صورت 
$X \sim Bernoulli(\frac{\pi}{4})$
تعریف می‌کنیم که در آن
$\pi$ و $\theta$
مجهول است.
نکته‌ای که در این سوال باید به آن توجه کنیم این است که
$\operatorname{Var}(X) = \frac{\pi}{4} (1 - \frac{\pi}{4})$
است. مشخص است که
$\max(\operatorname{Var}(X))$
زمانی اتفاق می‌افتد که
$\pi = 2$
باشد
(دقت کنید که در آزمایش نمی‌دانیم که
$\pi$
مساوی چند است و صرفا متغیری است که می‌خواهیم آنرا حدس بزنیم)
 که در این حالت
$\sigma^2 = \frac{1}{4}$
می‌شود.
پس بازه‌ی اطمینان در بدترین حالت که واریانس بیشینه باشد برابر است با:
\begin{gather*}
    \left[\bar{X}- z_{\frac{\alpha}{2}} \frac{\sigma_{max}}{\sqrt{n}} , \bar{X}+ z_{\frac{\alpha}{2}} \frac{\sigma_{max}}{\sqrt{n}}\right] = 
    \left[\bar{X}- z_{\frac{\alpha}{2}} \frac{\sqrt{\frac{1}{4}}}{\sqrt{n}} , \bar{X}+ z_{\frac{\alpha}{2}} \frac{\sqrt{\frac{1}{4}}}{\sqrt{n}}\right] =
    \left[\bar{X}- \frac{z_{\frac{\alpha}{2}}}{2\sqrt{n}} , \bar{X}+ \frac{z_{\frac{\alpha}{2}}}{2\sqrt{n}}\right]
\end{gather*}
حال طبق چیز‌های گفته شده باید
$\frac{z_{\frac{\alpha}{2}}}{2\sqrt{n}} = 0.0025$
باشد. همچنین دقت کنید که
$z_{\frac{\alpha}{2}} = z_{\frac{0.05}{2}} = z_{0.025} = 1.96$
است. پس:
\begin{align*}
    \frac{z_{\frac{\alpha}{2}}}{2\sqrt{n}} &= 0.0025\\
    \frac{1.96}{2\sqrt{n}} &= 0.0025\\
    \frac{0.98}{0.0025} &= \sqrt{n}\\
    392 &= \sqrt{n}\\
    153664 &= n
\end{align*}
