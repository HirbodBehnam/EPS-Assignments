\smalltitle{سوال 6}\\
فرض صفر این است که درصد دخترانی که استرس دارند با پسر‌ها فرقی ندارد.
فرض دیگر این است که این دو درصد تفاوت چشمگیری داشته باشند.
حال از
\lr{Welch's t-test}
استفاده می‌کنیم. فرض کنید که هر کسی که استرس بالا دارد به 1 نسبت می‌دهیم. در غیر این صورت به صفر نسبت می‌دهیم.
(عملا یک متغیر تصادفی تعریف کردیم.)
\begin{gather*}
    t = \frac{\overline{X}_1 - \overline{X}_2}{\sqrt{\frac{\sigma_1^2}{N_1} + \frac{\sigma_2^2}{N_2}}}\\
    \overline{X}_1 = \frac{18}{18 + 82} = 0.18  \quad \overline{X}_2 = \frac{38}{38 + 112} \approx 0.253\\
    \sigma_1^2 \approx 0.149 \quad \sigma_2^2 \approx 0.191\\
    t = \frac{|0.18 - 0.253|}{\sqrt{\frac{0.149}{100} + \frac{0.191}{150}}} = 1.388693\\
    df = \frac{\left( \; \frac{\sigma_1^2}{N_1} \; + \; \frac{\sigma_2^2}{N_2} \; \right)^2 }
    { \frac{\sigma_1^4}{N_1^2 (N_1 - 1)} \; + \; \frac{\sigma_2^4}{N_2^2 (N_2 - 1) }} = 229.04
\end{gather*}
حال به کمک قطعه کد زیر یا جدول‌های
\lr{t-table}
می‌توان
\lr{p-value}
را پیدا کرد: (دقت کنید که تست دو طرفه است)
\begin{latin}
\[ %https://stats.stackexchange.com/a/45156/359756
    p-value = \texttt{2*pt(1.388693, df = 229.04, lower=FALSE)} = 0.1662751
\]
\end{latin}
\noindent
از آنجایی که
$p-value$
عددی نسبتا بزرگ در آمد نشان می‌دهد که به احتمال خوبی نمی‌توان فرض صفر را رد کرد. از آنجایی که این عدد از
$0.05$ و $0.01$
بزرگتر است، پس نمی‌توان در قسمت‌های آخر سوال فرض را رد کرد.




