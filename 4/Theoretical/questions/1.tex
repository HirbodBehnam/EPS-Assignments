\smalltitle{سوال 1}\\
% https://www.lakeheadu.ca/sites/default/files/uploads/77/images/Sedor%20Kelly.pdf
فرض کنید که داریم:
\begin{gather*}
    \bar{X} = \frac{\sum_{i=1}^n X_i}{n}\\
    Y = \sum_{i=1}^n X_i
\end{gather*}
که در آن
$X_i$
یک
\lr{indicator variable}
است که زمانی یک می‌شود که سکه رو بیاید. آنگاه خواسته‌ی سوال برابر است با:
\begin{gather*}
    P(0.46 < \bar{X} < 0.5) > 0.95
\end{gather*}
دقت کنید که
$X_i \sim Bernoulli(0.48)$
است. حال مسئله را حل می‌کنیم.
همچنین یک سری از مشخصات مسئله به صورت زیر است:
\begin{gather*}
    \mathbb{E}(X_i) = 0.48 \quad \operatorname{Var}(X_i) = 0.48 \times 0.52 = 0.2496\\
    \mathbb{E}(Y) = n \mathbb{E}(X_i) = 0.48n \quad \operatorname{Var}(Y) = n\operatorname{Var}(X_i) = 0.2496n
\end{gather*}
\smalltitle{الف}
از نامساوی چویشف و قانون اعداد بزرگ داریم:
\begin{gather*}
    P(|\bar{X}_n - \mu| < \epsilon) \leq \frac{\sigma^2}{n\epsilon^2} \implies P(|\bar{X} - 0.48| < 0.02) \leq \frac{0.2496}{n(0.02)^2}\\
    \implies \frac{0.2496}{n(0.02)^2} = 0.05 \implies n = 12480
\end{gather*}
\smalltitle{ب}
\begin{gather*}
    P(0.46 < \bar{X} < 0.5) = P(0.46 < \frac{Y}{n} < 0.5) = P(0.46n < Y < 0.5n)\\
    = P(\frac{0.46n - 0.48n}{\sqrt{0.2496n}} < \frac{Y - 0.48n}{\sqrt{0.2496n}} < \frac{0.5n - 0.48n}{\sqrt{0.2496n}})\\
    = \Phi(\frac{0.02}{\sqrt{0.2496}}\sqrt{n}) - \Phi(-\frac{0.2}{\sqrt{0.2496}}\sqrt{n}) = 2\Phi(\frac{0.02}{\sqrt{0.2496}}\sqrt{n}) - 1\\
    \implies 2\Phi(\frac{0.02}{\sqrt{0.2496}}\sqrt{n}) - 1 \ge 0.95 \implies \Phi(\frac{0.02}{\sqrt{0.2496}}\sqrt{n}) \ge 0.975\\
    \implies n \ge 2397.07
\end{gather*}
پس باید سکه را
2398
بار پرتاب کنیم.
\\\smalltitle{ج}
مشخص است که قضیه‌ی حد مرکزی تعداد بار کمتری را مشخص می‌کند برای پرتاب سکه و برای همین بازه‌ی کوچکتری را به ما
می‌دهد و در نتیجه بهتر است.





