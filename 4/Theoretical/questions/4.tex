\smalltitle{سوال 4}\\
\smalltitle{الف}
% https://math.stackexchange.com/questions/1162965/find-the-expected-number-of-people-who-select-their-own-name-tag
\lr{indicator variable}
$X_i$
را در صورتی برابر یک قرار می‌دهیم که نفر
$i$ام
کفش خودش را انتخاب بکند. حال داریم:
\begin{gather*}
    \mathbb{E}(X) = \sum_{i=1}^n \mathbb{E}(X_i) = n (0 \times P(X_i = 0) + 1 \times P(X_i = 1))
    = n \times \frac{1}{n} = 1
\end{gather*}
\smalltitle{ب}
متغیر تصادفی
$X_{ij}$
را بدین صورت تعریف می‌کنیم که تنها زمانی یک شود که نفر
$i$ام
کفش نفر
$j$ام
و نفر
$j$ام
کفش نفر
$i$ام
را دریافت کرده باشد.
در این صورت داریم:
\begin{gather*}
    \mathbb{E}(X) = \frac{\sum_{i=1}^n \sum_{j=1}^{n (\neq i)} \mathbb{E}(X_{ij})}{2}\\
    \mathbb{E}(X_{ij}) = \frac{1}{n} \times \frac{1}{n-1}\\
    \implies \sum_{i=1}^n \sum_{j=1}^{n (\neq i)} \mathbb{E}(X_{ij}) = (n \times (n - 1)) \times (\frac{1}{n} \times \frac{1}{n-1}) = 1\\
    \implies \mathbb{E}(X) = \frac{1}{2}
\end{gather*}
دلیل اینکه نیاز به یک
$\frac{1}{2}$
داشتیم این بود که عملا
$X_{ij}$ با $X_{ji}$
فرقی ندارد و نباید دو بار شمرده شود. همچنین دقت کنید که
$X_{ii} = 0$
است. برای همین در سیگما داخلی نوشته‌ام که
$j \neq i$
باید باشد.
\\\smalltitle{ج}
فرض می‌کنیم که متغیر تصادفی
$X_i$
را طوری تعریف می‌کنیم که فقط زمانی یک باشد که
$a_i = b_i$
باشد. با این تعریف می‌توان
$\mathbb{E}(X)$
را به صورت
$\sum_{i=1}^n \mathbb{E}(X_i)$
نوشت.
$\mathbb{E}(X)$
همان امیدریاضی عناصر یونیک در مجموعه‌ی
$b_i$ها
می‌باشد. دلیل اینکه می‌توان
$\mathbb{E}(X)$
بدین گونه باز کرد این است که در صورتی که در
$b_i$
ماکسیموم برابر
$a_i$
نباشد، آنگاه ماکسیموم مثلا برابر
$a_j (j < i)$
است که در
$b_j$
آمده است پس این عدد یونیک نیست. حال خواسته‌ی مسئله را حساب می‌کنیم:
\begin{gather*}
    \mathbb{E}(X_i) = \frac{1}{i}\\
    \sum_{i=1}^n \mathbb{E}(X_i) = \sum_{i=1}^n \frac{1}{i}
\end{gather*}