\smalltitle{سوال 6}\\
باید از
\lr{pdf}
شرطی کمک بگیریم. می‌دانیم که
$f_{X|Y}(x|y) = \frac{f_{X,Y}(x,y)}{f_Y(y)}$
است. همچنین مشخص است که از آنجا که
$Y$
توزیع یونیفرم بین 0 و 1 دارد
$f_Y(y) = 1$
است. پس
$f_{X|Y}(x|y) = f_{X,Y}(x,y)$
است. حال باید سعی کنیم که
$f_{X|Y}(x|y)$
را بدست بیاوریم. برای این موضوع دقت کنید که اگر بدانیم که
$y$
چیست انگاه به راحتی می‌شود که
\lr{pdf}
توزیع
$(y, 1)$
را به کمک تعریف توزیع یونیفرم بدست آورد که برابر است با
$\frac{1}{1 - y}$.
پس داریم:
$f_{X|Y}(x|y) = \frac{1}{1 - y} = f_{X,Y}(x,y)$.
حال به کمک توزیع مارجینال
$f_X(x)$
را بدست می‌آوریم. دقت کنید که
$0 < y < x$
است.
\begin{gather*}
    f_X(x) = \int_0^x \frac{1}{1 - y} ~ dy = - \ln(1 - x) \quad 0 < x < 1
\end{gather*}