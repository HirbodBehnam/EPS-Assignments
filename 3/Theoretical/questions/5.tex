\smalltitle{سوال 5}\\
\smalltitle{الف}
در ابتدا
$\mathbb{E} [X^2]$
را بدست می‌آوریم طبق مطالب یاد گرفته شده در تمرین قبلی.
% http://cknudson.com/StudentWork/Transforms.pdf
% https://math.stackexchange.com/a/2131156/424863
\begin{gather*}
    f_X(x) = \frac{1}{2} \quad -1 < x < 1\\
    F_Y(y) = P(Y < y) = P(X^2 < y) = P(-\sqrt{y} < X < \sqrt{y}) = \int_{-\sqrt{y}}^{\sqrt{y}} f_X(x) ~ dx =
    \int_{-\sqrt{y}}^{\sqrt{y}} \frac{1}{2} ~ dx = \sqrt{y}\\
    \implies f_Y(y) = \frac{1}{2\sqrt{y}}\\
    \implies \mathbb{E} [Y] = \int_{0}^{1} y \times \frac{1}{2\sqrt{y}} ~ dy = \frac{1}{3}
\end{gather*}
دقت کنید که بازه‌ی انتگرال امیدریاضی به خاطر این 0 تا 1 شد چون
$-\sqrt{y} < X < \sqrt{y}$
بود و
$-1 < X < 1$
بود. پس
$0 < y < 1$
است.
\\\smalltitle{ب}
\begin{gather*}
    \mathbb{E} [g(X)] = \int_{-\infty}^{+\infty} g(x) f_X(x) ~ dx \implies
    \mathbb{E} [X^2] = \int_{-1}^{1} x^2 (\frac{1}{2}) ~ dx = \frac{1}{3}
\end{gather*}
\smalltitle{ج}
از آنجا که توزیع یونیفرم است پس
$\mathbb{E} [X] = 0$
است.
پس
$(\mathbb{E} [X])^2 \neq \mathbb{E} [X^2]$




