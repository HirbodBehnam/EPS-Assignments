\\\smalltitle{سوال 2}\\
زمانی که فرد آخر می‌خواهد بر روی صندلی خود بشیند دو حالت بیشتر ندارد. یا بر روی صندلی نفر اول
می‌نشنید و یا بر روی صندلی خود.
برای اثبات این موضوع از برهان خلف استفاده می‌کنیم. بدون کاستن از فرضیات فرض کنید که فرد
$i$ام
جای درستش بر روی صندلی
$i$ام
است.
 فرض می‌کنیم که در آخر صندلی خالی است که نه مال نفر آخر است و نه اول.
این نشان می‌دهد که در ابتدا یک نفر به صورت تصادفی رو روی صندلی نفر آخر نشسته است. این یعنی اینکه صندلی خودش
پر بوده است.
حال از نفر اول شروع به چیدن آدم‌ها می‌کنیم.
در صورتی که نفر اول به صورت تصادفی بر روی جای خودش نشسته باشد که همه سر جای خودشان می‌شینند و در نتیجه
نفر آخر نیز سر جای خودش می‌شیند. پس این حالت مورد قبول نیست.
\\~
فرض کنید نفر اول بر روی صندلی نفر آخر نشسته باشد. در این صورت همه بر روی جای خود می‌شینند به جز نفر آخر
که بر روی صندلی نفر اول می‌شیند که این نیز برخلاف فرض ماست.
\\~
فرض می‌کنیم که نفر اول به صورت تصادفی بر روی صندلی نفر
$i_1$
می‌نشیند.
پس نفرات
$2 < j < i_1$
بر روی جای خود می‌نشنید.
نفر
$i_1$
حال باید به صورت تصادفی یک جای دیگر را انتخاب کند. در صورتی که صندلی اول یا آخر را انتخاب کند که جای
نفر آخر صندلی اول یا آخر می‌شود. پس این فرد باید بر روی صندلی
$i_1 < i_2 < n$
بشنید که 
$n$
تعداد کل افراد است.
حال دقت کنید که در آخر فرض کنید که جای نفر
$n-1$ام
پر است. پس این فرد باید یا بر روی صندلی اول بشیند و یا آخر.
پس فرض خلف ما باطل بوده و نفر آخر حتما بر روی صندلی خودش می‌نشیند یا صندلی نفر اول.
\\~
حال دقت کنید که زمانی که یک مسافر بین صندلی آخر و اول یکی را انتخاب می‌کند، شانس انتخاب شدنشان برابر است.
این نشان می‌دهد که برای نفر آخر به احتمال
$\frac{1}{2}$
صندلی خودش خالی است و به احتمال
$\frac{1}{2}$
صندلی نفر اول. دقت کنید که در تمام طول محاسباتمان از تعداد افراد حرفی نزدیم پس قسمت الف و ب سوال با هم
جواب داده می‌شوند که جواب آن
$\frac{1}{2}$
است.
\\