\smalltitle{سوال 6}
\begin{enumerate}[wide, labelwidth=!, labelindent=0pt]
    \item عملا امیدریاضی شیر آمدن را باید حساب کنیم که به صورت
$\sum \frac{1}{2^i} \times \frac{1}{2^i} = \sum \frac{1}{4^i} = \frac{1}{3}$
است. این به خاطر این است که احتمال ساخته شدن سکه‌ای با احتمال شیر آمدن
$\frac{1}{2^i}$
برابر
$\frac{1}{2^i}$
است.
    \item با توجه به قسمت قبل در دو سکه‌ی متوالی احتمال شیر آمدن و سپس خط آمدن برابر است با
$\frac{1}{3} \times \frac{2}{3} = \frac{2}{9}$
(دقت کنید که احتمال شیر یا خط آمدن مستقل است)
حال فرض کنید که
$X$
را به
$\sum X_i$
می‌شکانیم که
$X_i$
نشان دهنده‌ی پول بدست آمده از پرتاب
$i$ و $i+1$ام
است. پس داریم:
\begin{gather*}
    \mathbb{E}[X] = \sum \mathbb{E}[X_i] = \sum_{i=1}^{n-1} \frac{2}{9} = \frac{2(n-1)}{9}
\end{gather*}
    \item در ابتدا باید
$\mathbb{E}[X^2]$
را حساب کنیم.
\begin{gather*}
    \mathbb{E}[X^2] = \mathbb{E}[ (\sum X_i) ^ 2] = \mathbb{E}[\sum X_i ^2 + 2 \sum \sum X_i X_j]
    = \sum \mathbb{E}[X_i ^2] + 2 \sum \sum \mathbb{E} [ X_i X_j ]
\end{gather*}
حال دقت کنید که
$\mathbb{E}[X_i ^2] = \mathbb{E}[X_i]$
است چرا که
$X_i$
تنها دو حالت 0 و 1 را دارد که اگر به توان دو نیز برسند تغییری نمی‌کنند.
(یا یک دلار می‌گیریم یا هیچی نمی‌گیریم.)
حال به تحلیل
$\mathbb{E} [ X_i X_j ]$
می‌پردازیم.
مشخص است که در صورتی که
$i$ و $j$
متوالی باشند، جواب آن‌ها باید صفر بشود چرا که امکان ندارد در دو پرتاب متوالی شیر، خط و شیر خط بیاید!
برای بقیه‌ی حالات نیز دقیقا با توجه به اصل ضرب
$\frac{2}{9} \times \frac{2}{9}$
جواب
$\mathbb{E} [ X_i X_j ]$
می‌شود. حال محاسبه می‌کنیم که به چند طریق می‌توان
$i$ و $j$
را انتخاب کرد. برای انتخاب خود آن‌ها
$n-1 \choose 2$
حالت وجود دارد که حالت‌های پشت سر هم قابل قبول نیست.
پس در کل
${n-1 \choose 2} - (n - 2)$
حالت قابل قبول وجود دارد.
در نهایت
$\mathbb{E} [ X ^2 ]$
را بدست می‌آوریم.
\begin{gather*}
    \mathbb{E} [ X^2 ] = \frac{2(n-1)}{9} + 2 (\frac{4}{81} ({n-1 \choose 2} - (n - 2)))
\end{gather*}
در نهایت واریانس را حساب می‌کنیم.
\begin{gather*}
    \operatorname{Var}(X) = \mathbb{E} [ X^2 ] - \mathbb{E} [ X ] ^2 = \frac{2(n-1)}{9} + \frac{8}{81} ({n-1 \choose 2} - (n - 2))
    - (\frac{2(n-1)}{9})^2
\end{gather*}
\end{enumerate}