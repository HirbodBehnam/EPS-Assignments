\smalltitle{سوال 4}\\
می‌دانیم که احتمال انتخاب شدن هر جایگشت یکسان است در کل
$n!$
جایگشت داریم. پس
$\mathbb{E}[X] = \frac{x}{n!}$
که $x$ مجموع پولی است که از تمامی جایگشت‌ها می‌توانیم بدست آوریم.
حال به محاسبه‌ی
$x$
می‌پردازیم.

برای محاسبه‌ی
$x$
بدین صورت عمل می‌کنیم که برای هر عدد مانند
$k$
حساب می‌کنیم که در چند حالت از
$n!$
حالت از تمامی اعداد قبل خود بزرگتر است.
(فرض میکنیم که در صورتی که عدد اول بیاید همیشه باعث گرفتن پول از پدر علی می‌شود.)
در ابتدا باید جایگاهی برای
$k$
تعیین کنیم که آن‌را در آن جایگاه بگذاریم. دقت کنید که
$k$
را فقط در جایگاه‌های
$1$
تا
$k$
می‌توانیم بگذاریم؛ چرا که در صورتی که اگر در جایگاه‌های بزرگتر از آن قرار دهیم، مجبوریم که عددی بزرگتر از
$k$
را در یکی از جایگاه‌های قبلش قرار دهیم.

حال فرض کنید که جایگاه
$i$ام
را انتخاب کرده‌ایم. (جایگاه‌ها از یک شروع می‌شوند.)
در ابتدا باید که از بین
$k-1$
اعداد کوچکتر از $k$،
$i-1$
عدد را انتخاب کنیم که در جایگاه‌های قبل از
$k$
قرار دهیمشان. سپس برای آن‌ها
$(i-1)!$
جایگشت داریم. سپس باید
$n-i$
عدد باقی‌مانده را در
$(n-i)!$
جایگشت مختلف چید.
پس درنهایت برای هر
$k$
داریم:
\begin{gather*}
    \sum^k_{i=1} {k-1 \choose i-1} \times (i - 1)! \times (n - i)! = 
    \sum^k_{i=1} P^{k-1}_{i - 1} \times (n - i)!
\end{gather*}
در نهایت باید برای تمامی
$k$ها
این عبارت را حساب کنیم. پس در نهایت:
\begin{gather*}
    x = \sum^n_{k=1} \sum^k_{i=1} P^{k-1}_{i - 1} \times (n - i)!
\end{gather*}
\begin{gather*}
    \mathbb{E}[x] = \frac{\sum^n_{k=1} \sum^k_{i=1} P^{k-1}_{i - 1} \times (n - i)!}{n!}
\end{gather*}